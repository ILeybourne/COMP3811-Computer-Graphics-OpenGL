\chapter{Ethics}
\label{ch:3ethics}

\section{Ethical concerns}
\label{sec:ethicalconcerns}
\subsection{Offensive Images}
\label{subsec:offensiveimages}
There is a wide range of ethical concerns to consider for this project. As the image generation process is automated in this project by the model, it is important to be mindful of how the output images generated could be considered distasteful or offensive to some individuals. Earlier this year Antonio Torralba, Rob Fergus and Bill Freeman, researchers at the Massachusetts Institute of Technology, announced that they had decided to remove a dataset named TinyImages due to its usage of vulgar labels and offensive images. The dataset was a large dataset with 80 million images, each with the resolution of 32 by 32 pixels, that was automatically generated by web scraping. This aspect made it extremely difficult to manually remove offensive material leading to the team decided it was better to remove the dataset in its entirety \cite{tinyimages}. This leads us to believe that we should exercise caution when choosing the datasets that will be used to train and test my model, given that some creators of particular datasets may have not exercised the best judgement when constructing their datasets from the data they had at their disposal. As it is not always possible to manually inspect a dataset (due to restrictions of resources such as time and manpower) it is best to take use dataset that come from reliable sources and have some understanding of how they were created. Failure to do this can lead to distasteful or obscene materials used to create the model, which in turn could lead to the model outputting obscene images. 

\subsection{Unintentional Outputs}
\label{unintentionaloutputs}
\noindent Although not explicitly relevant to image generation, Microsoft’s Tay – an AI launched on Twitter in 2016 marketed as a chat-bot – received mass criticism from the public and journalists alike due Tay’s usage of racist and sexist slurs and various other insulting remarks made on the social media network. After two days Microsoft removed the AI from the platform and released an apology stating that “a coordinated attack by a subset of people exploited a vulnerability in Tay”\cite{microsofttay}. Although this project is not designed to be a user on such an open platform like Twitter, this example still serves as a demonstration that not all end users will use the software presented in this project as intended and may seek to exploit bugs within the program, which would lead to a poor reflection on my work and possible perceived affiliations with insulting material.

\subsection{Copyright and Intellectual Property}
\label{subsec:copyrightandintellectualpropery}
When planning the implementation of this project care was exercised in ensure no Copyright infringement took place with regard to software, APIs, and resources used. In addition, plagiarism was avoided at all cost and all usages of authors work has been cited see \hyperref[refs]{'References'}.

\subsection{Social Implications}
\label{subsec:socialimplications}
Automation can lead to rapid change in job markets and otherwise change people's lives significantly. The scope of this project is too small to drastically make any meaningful impact on a societal level.

\subsection{Privacy Concerns}
\label{subsec:privacyconcerns}
This project will be using images of textures using under the creative commons or with the permission of the creators. No infringement on people's human right to privacy will take place as no image of people will be used.